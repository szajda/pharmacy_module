\chapter{\noun{Database Functions and Schema}}

\section{\noun{Database Functions}}
After veryfing credentials sent by user to Central Server via SSL secure channel, user, depending on its role, will be able to execute set of functions, which will serve single purpose each (e.g. creation of new prescription).

\subsection{\noun{Shared Functions}}
%Browse medicines
\small
    \begin{longtable}{| p{3cm} | p{10.75cm} |}
    \hline
     & browse\_medicines \\ \hline
    Arguments: &  \begin{packed_enum}
    \item name (string, optional, default = None)
    \item type (string, optional, default = None)
	\end{packed_enum}        \\ \hline
    Usage: & Pharmacist sends his id and id of current patient which are signed by both of their keys. Pharmacist can see only prescriptions which are not yet realized. \\ \hline
    Result: & \begin{packed_enum}
    	\item medicine\_id
    	\item name
    	\item prescription requirement
    	\item medicine type
    	\item maximum dosage
    	\item unit
	\end{packed_enum}     \\ \hline	
    \end{longtable}

Note: Multiple records may be returned at single request.
%Browse doctors
    \begin{longtable}{| p{3cm} | p{10.75cm} |}
    \hline
     & browse\_doctors \\ \hline
    Arguments: &  \begin{packed_enum}
    	\item name ( string, optional, default = None)
		\item address (string, optional, default = None)
		\item license\_number (string, optional, default = None)

	\end{packed_enum}     \\ \hline
    Usage: & Entity using this function performs simple query which return all public data regarding registered doctors stored in DB. Arguments name, adress and license\_number narrows down result applying filters to the executed query. \\ \hline
    Result: & \begin{packed_enum}
    	\item doctor\_id
		\item name
		\item address
		\item license\_number
		\item certificate
		\item public\_key
	\end{packed_enum}     \\ \hline	
    \end{longtable}

Note: Multiple records may be returned at single request.

\newpage
%Browse Pharmacies

    \begin{longtable}{| p{3cm} | p{10.75cm} |}
    \hline
     & browse\_pharmacists \\ \hline
    Arguments: &  \begin{packed_enum}
    	\item pharmacist\_name ( string, optional, default = None)
		\item address (string, optional, default = None)
		\item license\_number (string, optional, default = None)
		\item pharmacy\_name (string, optional, default = None)

	\end{packed_enum}     \\ \hline
    Usage: & Entity using this function performs simple query which return all public data regarding registered pharmacists stored in DB. Arguments name, adress and license\_number narrows down result applying filters to the executed query. \\ \hline
    Result: & \begin{packed_enum}
    	\item pharmacist\_id
		\item name
		\item address
		\item license\_number
		\item certificate
		\item public\_key
		\item pharmacy\_name
	\end{packed_enum}     \\ \hline	
    \end{longtable}
Note: Multiple records may be returned at single request.

\subsection{\noun{Patient Functions}}

%GetPatientNonce

    \begin{longtable}{| p{3cm} | p{10.75cm} |}
    \hline
     & get\_patient\_nonce \\ \hline
    Arguments: &  \begin{packed_enum}
    	\item patient\_id (integer, mandatory)
	\end{packed_enum}     \\ \hline
    Usage: & Function returns 1024 bit nonce for given patient\_id. \\ \hline
    Result: & \begin{packed_enum}
    	\item nonce
	\end{packed_enum}     \\ \hline	
			Comment: & New nonce is generated only if the last request was successfully verified.\\ \hline
    \end{longtable}


%Browse My Prescriptions History

    \begin{longtable}{| p{3cm} | p{10.75cm} |}
    \hline
     & browse\_my\_prescriptions\_history \\ \hline
    Arguments: &  \begin{packed_enum}
    	\item patient\_id (integer, mandatory)
		\item executed (boolean, optional, default = None)
		\item start (date, optional, default = None)
		\item end (date, optional, default = None)
		\item patient\_signature (byte, mandatory)
	\end{packed_enum}     \\ \hline
    Usage: & Patient requires history of his prescriptions. In order to get access to this kind of data, patient needs to sign his request using his secret key. Next, the signature will be veryfied by database. If signature will be acknowledged as genuine, database will return data about patient prescription history. Database provides patient the ability to filter his history by mean of time span and by the information about execution of prescriptions. \\ \hline
    Result: & \begin{packed_enum}
    	\item prescription\_id
    	\item doctor\_id
    	\item doctor name
    	\item doctor address
    	\item doctor license number
    	\item prescription\_owner\_id
    	\item drug id
    	\item dosage
    	\item max dosage
    	\item unit
    	\item quantity
    	\item execution
    	\item time of execution
    	\item pharmacy\_id
    	\item pharmacy\_name
    	\item pharmacy\_adress
	\end{packed_enum}     \\ \hline	
    \end{longtable}
Note: Multiple records may be returned at single request.

\newpage
%transfer_prescription

    \begin{longtable}{| p{3cm} | p{10.75cm} |}
    \hline
     & transfer\_prescription\\ \hline
    Arguments: &  \begin{packed_enum}
    	\item patient\_id (integer, mandatory)
    	\item owner\_PESEL (integer, mandatory)
		\item prescription\_id (integer, mandatory)
		\item patient\_signature (byte, mandatory)
	\end{packed_enum}     \\ \hline
    Usage: & Patient changes prescription owner to another patient, therefore allowing him to buy out specific prescription. It is important note, that after changing owner of prescription, original owner is NOT able to buy out his prescription until transfer is cancelled. \\ \hline
    Result: & \begin{packed_enum}
    	\item new\_owner\_id
		\item "OK"

	\end{packed_enum}     \\ \hline	
		Comment: & Prescription in database structure has two fields indicating prescription ownership - patientID (non-changeable, indicates the patient to which the medicine was prescribed) and owner\_PESEL (patient which will may realize the prescription).
		Transfering the right will only apply if both of these fields point to same id - thus we exclude the scenario when patients can pass the prescription to yet another person. After this operation the transferring patient losses right to realize the prescription - prevention from cloning the prescription.\\ \hline
    \end{longtable}


%Cancel prescription transfer

    \begin{longtable}{| p{3cm} | p{10.75cm} |}
    \hline
     & cancel\_prescription\_transfer \\ \hline
    Arguments: &  \begin{packed_enum}
    	\item patient\_id (integer, mandatory)
		\item prescription\_id (integer, mandatory)
		\item patient\_signature (byte, mandatory)
	\end{packed_enum}     \\ \hline
    Usage: & Patient changes actual owner of his prescription back to the original one (the patient himself) allowing him to buy out prescription and disallowing former owner of prescription to do so. \\ \hline
    Result: & \begin{packed_enum}
    	\item "OK"
	\end{packed_enum}     \\ \hline	
		Comment: & Prescription in database structure has two fields indicating prescription ownership - patientID (non-changeable, indicates the patient to which the medicine was prescribed) and ownerID (patient which will may realize the prescription). Cancelling will only work if patientID and ownerID are different and the signature over request is verified. \\ \hline
    \end{longtable}

\subsection{\noun{Doctor Functions}}

%get_doctor_nonce

    \begin{longtable}{| p{3cm} | p{10.75cm} |}
    \hline
     & get\_doctor\_nonce \\ \hline
    Arguments: &  \begin{packed_enum}
    	\item doctor\_id (integer, mandatory)
	\end{packed_enum}     \\ \hline
    Usage: & Function returns 1024 bit nonce for given doctor\_id. \\ \hline
    Result: & \begin{packed_enum}
    	\item nonce
	\end{packed_enum}     \\ \hline	
			Comment: & New nonce is generated only if the last request was successfully verified.\\ \hline
    \end{longtable}


%create_prescription

    \begin{longtable}{| p{3cm} | p{10.75cm} |}
    \hline
     & create\_prescription \\ \hline
    Arguments: &  \begin{packed_enum}
    	\item doctor\_id (integer, mandatory)
		\item patient\_id (integer, mandatory)
		\item drug\_id (integer, mandatory)
		\item dosage (integer, mandatory)
		\item unit (integer, mandatory)
		\item quantity (integer, mandatory)
		\item doctor\_signature (byte, mandatory)
	\end{packed_enum}     \\ \hline
    Usage: & Doctor prescribe single medicine to the patient, describing medicine, quantity and dosage. \\ \hline
    Result: & \begin{packed_enum}
    	\item "OK"
	\end{packed_enum}     \\ \hline	
		Comment: & Database does not requires patient signature to create a prescription for him - Prescription realization will require his key (thus his smartcard) so the medicine can't be bought without his knowledge. Also the doctor can create prescription without the need of meeting the patient face to face - which is and advantage for chronically ill patients.\\ \hline
    \end{longtable}


%browse_patient_prescription_history

    \begin{longtable}{| p{3cm} | p{10.75cm} |}
    \hline
     & browse\_patient\_prescription\_history \\ \hline
    Arguments: &  \begin{packed_enum}
    	\item doctor\_id (integer, mandatory)
		\item patient\_id (integer, mandatory)
		\item start (date, optional, default = None)
		\item end (date, optional, default = None)
		\item bought(boolean, optional, default = None)
		\item doctor\_signature (byte, mandatory)
		\item patient\_signature (byte, optional)

	\end{packed_enum}     \\ \hline
    Usage: & Doctor downloads patient prescription history. Doctor (unlike pharmacist) do not needs patient signature to browse history od prescription that he has created. If he wants the full history, patients signature is needed. \\ \hline
    Result: & \begin{packed_enum}
    	\item prescription\_id
    	\item doctor\_id
    	\item doctor name
    	\item doctor address
    	\item doctor license number
    	\item prescription\_owner\_id
    	\item drug id
    	\item dosage
    	\item max dosage
    	\item unit
    	\item quantity
    	\item execution
    	\item time of execution
    	\item pharmacy\_id
    	\item pharmacy\_name
    	\item pharmacy\_adress
	\end{packed_enum}     \\ \hline
	Comment: & If the patient signature is missing, database will only return prescriptions which were created by the doctor. If the patient signature is present and can be verified, doctor will receive the full history of patient. In case of any errors on verification, the request will be canceled. \\ \hline
    \end{longtable}
Note: Multiple records may be returned at single request.

\subsection{\noun{Pharmacist functions}}

%get_pharmacist_nonce

    \begin{longtable}{| p{3cm} | p{10.75cm} |}
    \hline
     & get\_pharmacist\_nonce \\ \hline
    Arguments: &  \begin{packed_enum}
    	\item pharmacist\_id (integer, mandatory)
	\end{packed_enum}     \\ \hline
    Usage: & Function returns 1024 bit nonce for given pharmacist\_id. \\ \hline
    Result: & \begin{packed_enum}
    	\item nonce
	\end{packed_enum}     \\ \hline	
			Comment: & New nonce is generated only if the last request was successfully verified.\\ \hline
    \end{longtable}

%prescription_realization

    \begin{longtable}{| p{3cm} | p{10.75cm} |}
    \hline
     & prescription\_realization \\ \hline
    Arguments: &  \begin{packed_enum}
    	\item prescription\_id (integer, mandatory)
		\item pharmacist\_id (integer, mandatory)
		\item drug\_id (integer, mandatory)
		\item unit (integer, mandatory)
		\item quantity (integer, mandatory)
		\item pharmacist\_signature (byte, mandatory)
		\item patient\_signature (byte, mandatory)

	\end{packed_enum}     \\ \hline
    Usage: & Pharmacist will be able to realize patient prescription by pointing right prescription by giving its id, choose proper medicine (not necessairly the same as medicine prescribed by doctor, this check will be done by database), describe how many medicine is sold.\\ \hline
    Result: & \begin{packed_enum}
    	\item "OK"
	\end{packed_enum}     \\ \hline	
	Comment: & Request has to be signed by both patient's and pharmacist's keys. If the signature is incorrect, the database will return error message and the medicine shouldn't be given away.\\ \hline
    \end{longtable}


%browse_active_prescriptions

    \begin{longtable}{| p{3cm} | p{10.75cm} |}
    \hline
     & browse\_active\_prescriptions \\ \hline
    Arguments: &  \begin{packed_enum}
    	\item pharmacist\_id (integer, mandatory)
		\item patient\_id (integer, mandatory)
		\item pharmacist\_signature (byte, mandatory)
		\item patient\_signature (byte, mandatory)
	\end{packed_enum}     \\ \hline
    Usage: & Pharmacy is able to see all active (not bought) prescriptions of current patient, which agrees to show this data to the pharmacy by signing request. \\ \hline
    Result: & \begin{packed_enum}
    	\item prescription\_id
    	\item doctor\_id
    	\item doctor name
    	\item doctor address
    	\item doctor license number
    	\item prescription\_owner\_id
    	\item drug id
    	\item dosage
    	\item max dosage
    	\item unit
    	\item quantity
    	\item execution
    	\item time of execution
	\end{packed_enum}     \\ \hline	
		Comment: & If the signatures of patient or pharmacist are incorrect, database will return an error. If there are no non-realized prescriptions, database will return empty list.\\ \hline
    \end{longtable}
Note: Multiple records may be returned at single request.
\normalsize
\newpage
\section{\noun{Database schema}}

\begin{figure}[h]
\begin{center}
\includegraphics[width=1\textwidth , angle=90]{database/databaseSchema.png}
\end{center}
\end{figure} 

\chapter{\noun{Project}}


\section{\noun{The Main Objectives}}

The main objectives of our new design of the pharmacy module is to limit the impact of the threats listed above and improve the usability of the current system.
\newline
\textbf{ The patient} has to be sure that his sensitive data is stored in a secure way, and unauthorized person cannot get to know anything about his medicines and illnesses. 
\newline
\textbf{The pharmacist} has to be sure that he sells the right medicines only for the right patient. 
\newline
\textbf{The refund process} should be quicker and easier. 
\newline
\textbf{The} possibility of making \textbf{mistakes} on the prescription should be eliminated. 
\newline
\textbf{The} number of \textbf{defraudations} should be significantly limited.


\section{\noun{Environment Requirements}}

\subsection{\noun{Smart Cards}}

The main reason we decided to use smart cards is that smart card solutions, which employs two factor authentication, i.e. "something you have and something you know", provide a high security level which is crucial for the health's systems sensitive data.

All the system's users will be given personalized smart cards which will store their identification data: names, surnames, PESEL and digital certificates. Each card will be assigned PIN and PUK numbers. The first one will be used to initialize authentication process, the second one will be used for unblocking a card\footnote{Unblocking procedure can be performed in the two following situation: when a user inputs wrong PIN number three times in a row or when he blocks his card after loosing it.}. 

To improve the security level of the system, the data  stored on smart cards should be enciphered. Users' private keys need to be stored in a secure memory which cannot be directly read out. 

In case of loosing a smart card, a user should perform a standardized revocation procedure. First, he should block a card in the assigned institution and while doing this he should be able to select whether he want to block the card temporarily or permanently. In the first case, after finding the card it is possible to unblock it with card's PUK number. In the second case it is necessary to generate new user's card and even after finding the card it will not be possible to unblock it.


\subsection{\noun{Certificates}}

Each user has his digital certificate on his smart card. All the user's certificates must be given by a defined certification authority and regularly\footnote{The CA should define a standard validity period for the patient's, pharmacist's and doctor's certificates.} updated.

In case of selecting permanent blocking option during the revocation procedure, a new certificate is generated for such user.

The certificate's validity should be checked at each use of the user's smartcard. The validity check is performed in the database module.

\subsection{\noun{Pharmacy}}

All the pharmacies which will be using the system must have broadband internet access, two smart card readers and two terminals: one for a pharmacist and one for a customer. The terminals apart from displaying the data need to handle all the confirmation actions on both sides. \newline


\section{\noun{Architecture}}

The pharmacy module architecture consists of the following elements:
\begin{enumerate}
\item \textbf{smart cards} with personal certificate, used for the authentication and signing, and an application which allows to read certain data from the card;
\item \textbf{pharmacist's PC} with a pharmacy module application which provides all of the functionalities which satisfy all the operation performed in a pharmacy; provides two user-friendly interfaces: one for a patient and one for a pharmacist; is connected with patient’s and pharmacist’s terminals and the central database; is able to execute SIGMA protocol, handle secure keys storage and establish SSL connection;
\item{\textbf{central DB}} is a central element of the whole system; stores the data and handles all the necessary database I/O functions.
\end{enumerate}

\section{\noun{Protection and Security}}

Below we describe entities used in the system, how we choose to protect them and why.

\subsection{\noun{Protection methods}}

\subsubsection{\noun{Patient's card}}

Patient's card stores private key along with the certificate. Elements of the certificate are as follows ({\it text in parentheses describes what is used}):
\begin{itemize}
 \item {\bf Serial Number}: Used to uniquely identify the certificate.
 \item {\bf Subject}: The person, or entity identified ({\it personal data of the patient}).
 \item {\bf Signature Algorithm}: The algorithm used to create the signature ({\it RSA}).
 \item {\bf Signature}: The actual signature to verify that it came from the issuer.
 \item {\bf Issuer}: The entity that verified the information and issued the certificate ({\it CA for the patient}).
 \item {\bf Valid-From}: The date the certificate is first valid from.
 \item {\bf Valid-To}: The expiration date.
 \item {\bf Public Key}: The public key.
 \item {\bf Thumbprint Algorithm}: The algorithm used to hash the public key certificate ({\it SHA256}).
 \item {\bf Thumbprint (also known as fingerprint)}: The hash itself, used as an abbreviated form of the public key certificate.
\end{itemize}

\subsubsection{\noun{Pharmacist's card}}

Pharmacist's card stores the same information as patient's card, with exception to several certificate fields being different:
\begin{itemize}
 \item {\bf Subject:} {\it Personal data of the pharmacist and pharmacy}
 \item {\bf Issuer:} {\it CA for the pharmacies}
\end{itemize}

\subsubsection{\noun{Card's data access}}

Card is read only is the sense that patients/pharmacists are not able to modify the data that is stored on it. They do, however (after successful authentication), have access to certificate stored on the card as well as the function to sign arbitrary input data with its private key.

\subsubsection{\noun{PIN}}

The certificate access/signing input data can be performed after inputting a PIN. The user is given 4-digit PIN number and the verification system will allow three attempts of typing the correct number before the card is blocked.

\subsubsection{\noun{Patient authentication}}

Two factor authentication is used:
\begin{itemize}
 \item Something you have - smart card (containing user's certificate)
 \item Something you know - PIN number used to access the certificate on the card
\end{itemize}

\subsubsection{\noun{Pharmacist authentication}}

Two factor authentication is used:
\begin{itemize}
 \item Something you have - smart card (containing pharmacist's certificate)
 \item Something you know - PIN number used to access the certificate on the card
\end{itemize}

\subsubsection{\noun{Connection between card and application}}
After successful authentication we establish a session key and the communication is encrypted with it. For that AKE protocol ,,SIGMA'' is utilized. (Note that encryption is optional if we assume that no eavesdropping can take place or the data exchanged isn't considered confidential).

\subsubsection{\noun{Connection between application and central database}}
Two-way SSL connection is used along with additional nonce-based authentication.

\subsection{\noun{Justification}}

\subsubsection{\noun{PIN protection}}
The PIN number is used to authenticate the card holder. In case the card was lost and found by someone else, he won't be able to use the card without knowing the PIN. We propose 4-digit PIN number with three subsequent incorrect attempts before the card is blocked as it's already used e.g. in ATM cards and proven to work there.

\subsubsection{\noun{SIGMA protocol}}
This AKE protocol (we choose to use SIGMA, but that is by no means the ultimate choice. It's been chosen due to convenience of having the implementation already in place. If one wishes, it can be replaced by other AKE protocol, e.g. NAXOS) will be used to secure the communication channels between parties existing in the pharmacy, i.e. cards and application. AKE protocols provide not only secure communication but also authentication mechanism, preventing not only eavesdropping or man-in-the-middle attacks but also party substitution.

\subsubsection{\noun{Secure key disposal}}
All short term keys, i.e. ephemeral keys used using AKE protocol or session keys which are the result of the protocol are erased from memory immediately after they are no longer needed.

\subsubsection{\noun{Secure communication with the database}}
The communication channel between database and pharmacy is secured with an SSL connection. We assume the SSL provides all the necessary mechanisms to protect the channel from attacks. To strengthen the security of the channel all the requests from any valid party must contain the signature (RSA signature) over the nonce provided by database system. This solution ensure that no unauthorized party is able to get access to database.

\subsubsection{\noun{Protection against defraudation}}
Each transaction has to be signed by all the participating parties. In this setting it is impossible for the doctor/pharmacist to fake the medicaments sale and deceive NFZ into giving them money for refunding the nonexistent costs, as signature of the patient is also required. Additionally a token that has to be signed changes with each transaction, so protection from repetition attacks is also gained.
